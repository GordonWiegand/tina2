%\renewcommand*\chappic{./bilder/160120_uhr.png}
\includegraphics[height=.8\textheight]{./bilder/160120_uhr.png}
\vskip 1cm
{\Huge\color{blue}\hfill\textsf{\textbf{Die Uhr}}}
\addcontentsline{toc}{chapter}{Die Uhr}
\newpage
%\chapter*{Die Uhr}
\lettrine[lines=2, lhang=.2, loversize=.25, lraise=0.05, findent=0.1em, nindent=0em]{W}{}enn jemand nach der Zeit fragt, will die Person wissen,
wie spät es ist. Ausser die Person ist eine Physikerin oder ein Philosoph, dann
kann es sein, dass sie etwas anderes meinen, jedenfalls wenn sie im Dienst sind
und nicht gerade auf den Bus warten.

Aber was ist das eigentlich, \textit{Zeit}? Das weiss leider niemand so genau
und ist eines der grossen Rätsel. Angenommen, jemand wüsste, was Zeit ist, dann
müsste die Person auch sagen können, was \textit{keine Zeit} ist. Das ist aber
irgendwie unmöglich. Das würde man ja gar nicht merken können. Wenn die Zeit
stehen bliebe, würde das niemand merken können, da für alle und alles die Zeit
stehen bleiben würde. Also ist die Zeit nie stehen geblieben, die Dauer, die
Zeit stehen bleiben kann, ist immer genau Null Sekunden. Zeit bleibt also nie
stehen, Zeit ist immer.

Aber wir merken, dass die Zeit vergeht. Mal scheint sie schnell, mal langsam zu
vergehen. Vergleiche einmal die gleichen zehn Minuten, die du noch aufbleiben
darfst, bis du ins Bett musst mit den 10 Minuten, die du im Zimmer bleiben
musst kurz vor der Bescherung zu Weihnachten.

Aber woran merken wir, dass Zeit vergeht. Wenn man die Physikerin fragt, wird
sie sagen, dass wir Zeit nur durch Bewegung wahrnehmen können. Etwas hast sich
bewegt, also ist Zeit vergangen. Zeit und Bewegung sind sehr nah verwandt und
die Physikerin wird dir erklären, dass beides sehr, sehr eng zusammengehört und
nur irgendwie dasselbe ist.

Das Zeit und Bewegung austauschbar sind, haben Menschen schon sehr früh gemerkt
und sich nutzbar gemacht, um das eine mit dem anderen zu messen. Bei einem
Wettrennen wollen wir wissen, wer die Schnellst ist. Dafür stoppen wir die
Zeit. Wenn wir die Zeit messen wollen, beobachten wir im Gegensatz immer etwas,
dass sich bewegt.

Das älteste Beispiel ist die Sonne. Auch wenn die sich nur scheinbar bewegt und
sich eigentlich die Erde dreht, scheint sie am Morgen aufzugehen, am Mittag
steht sie ziemlich weit oben am Himmel und dann bewegt sie sich weiter in
Richtung Sonnenuntergang. Das ist aber manchmal nicht genau genug. Ein bisschen
besser wurde es, als sie einen Stock in die Erde gesteckt haben und den
Schatten, den er wirft, verfolgt haben. Auch ein Kalender ist nichts anderes
als ein altes Zeitmessgerät.

Vermutlich würde die Genauigkeit von so einer Sonnenuhr für viele Zwecke in
deinem Alltag genügen. Aber leider scheint die Sonne ja nicht jeden Tag und
nachts schon gleich gar nicht und ob so eine Sonnenuhr anzeigen kann, ob es
der Physikerin noch genügt einen Kaffee zu trinken, bis der Bus kommt, glaube
ich auch nicht.

Das Zusammenleben von uns Menschen ist ja sehr darauf angewiesen, dass wir die
Zeit gut messen können. Um zehn Minuten nach acht Uhr geht die Schule los, also so musst Du um
sieben Uhr aufstehen, gegen halb acht musst du mit dem Frühstück fertig sein,
dich dann anziehen und so gegen dreiviertel acht loslaufen. 

Um so einen Plan hin zu bekommen, benötigen wir ein besseres Gerät, um die Zeit zu messen. Der Trick ist,
dass das etwas sein muss, was sich möglichst gleichmässig bewegt und diese
Bewegung mindestens einen Tag lang beibehält. Die Menschen haben da so einiges
ausprobiert. Sie haben Markierungen auf Kerzen gemalt und dann nachgesehen, wie
weit die Kerze schon abgebrannt ist, sie haben Zündschnüre angezündet und
beobachtet, wie weit die schon verbrannt sind und vieles andere mehr. Aber so
richtig zufriedenstellend war das nie.

Die vielleicht wichtigste Erfindung in der Geschichte der Uhr war die
\textit{Hemmung}. Eine Hemmung ist das Herzstück einer Uhr mit Zahnrädern.
irgendwo her kommt eine Kraft, beispielsweise durch ein Gewicht, durch Wasser
oder in beispielsweise Armbanduhren eine Feder. Und die Kraft wird benutzt, um
die Zeiger einer Uhr, so wie du sie kennst zu bewegen. Aber diese Kraft ist oft
nicht so gleichmässig, dass die Zeiger sich auch immer mit derselben
Geschwindigkeit bewegen. Das kann aber eben diese Hemmung bewirken. Wen nDu mal
eine alte Standuhr siehst oder eine Armbanduhr mit Zeigern siehst, hör mal
genau hin: das Tick-Tack, Tick-Tack, Tick-Tack ist die Hemmung. 

So eine Räderuhr war ein riesiger Fortschritt. Jetzt war es endlich möglich,
dass alle eine gemeinsame Zeit messen konnten. Sehr präzise waren diese Uhren
zwar anfänglich nicht, aber der Fortschritt war gewaltig.

Wenn du beispielsweise an die ersten Seefahrer denkst, die sich weit aufs
offene Meer hinau getraut haben, was meinst du, was für die wichtig war, um zu
überleben? Genau, die mussten wissen, wo sie sind, also wie weit sie schon
gefahren sind, sie mussten also Bewegung messen. Und dafür mussten sie eben die
genaue Zeit wissen. Weil die Uhren noch ungenau waren, hatten Schiffe früher
manchmal einen eigenen Raum, der nur mit Uhren gefüllt war. Die
durchschnittliche Zeit aller Uhren war dann meistens ein schon ziemlich
präziser Zeitmesser.

Genauere Uhren haben also viel bewirkt und helfen uns bei unserem Zusammenleben
und sind seitdem immer wichtiger geworden. Also wurden auch immer genauere
Uhren benötigt. Die Bewegung, die moderne Uhren nutzen, sind zum Beispiel das
schwingen von Quarzen, die unter Strom stehen. Es lässt sich sogar das
Schwingen von Atomen messen. Das geschieht mit Hilfe sehr grosser Uhren in
Forschungslaboren. Und andere Uhren, beispielsweise die am Computer, in
Telefonen oder die Funkuhr an der Wand können sich nach diesen sehr genauen
Uhren richten. 

Das die Uhren so genau geworden sind, hat einen grossen Einfluss auf unser
Leben gehabt und hat es weiterhin. Ein Zugfahrplan wäre ohne minutengenaue
Uhren nicht denkbar. Eine Eieruhr wäre ohne sekundengenaue Uhr nicht möglich.
Im Sport benötigen wir Uhren, die den hundersten und sogar tausendsten Teil
einer Sekunde genau messen können müssen. Und Uhren die noch genauer sind,
können die Zeit messen, die ein Impuls von deinem Telefon zu einem Satelliten
und zurück benötigt, woraus dein Telefon deinen genauen Standort berechnen
kann. 

Genaue Zeitmessung ist eine grossartige Herausforderung für Ingenieure und
unsere Physikerin. Sie ermöglicht so viel und immer grössere Genauigkeit zu
erreichen ist etwas, woran sehr viele Menschen arbeiten. Manche, wie die
Uhrmacher im Jura, haben die Herstellung von mechanischen Uhren zur Kunstform
gemacht. Sie versuchen nur durch Mechanik eine hohe Präzision zu erreichen. 

Aber was genau das ist, was diese Uhren, ob schön oder sehr genau messen, das
weiss noch immer niemand so richtig.

%Chrnopsychologie
%Philosophie der Zeit
%Bewegung
%Mechanik

