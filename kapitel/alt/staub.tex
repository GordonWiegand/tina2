\thispagestyle{empty}
\begin{center}
%\includegraphics[height=.8\textheight]{./bilder/160120_uhr.png}
\end{center}
\vskip 2cm
%{\Large\color{farbe}\hfill{\merriweathersansblack Über den Staubx}}
{\Large\color{farbe}\hfill{\merriweathersans \textbf{Über den Staub}}}
\addcontentsline{toc}{chapter}{}
\newpage
%%%%%%%%%%%%%%%%%%%%%%%%%%%%%%%%%%%%%%%%%%%%%%%%%%%%%%%%%%%%%%%%%%%%%%%%%%%%%%%
\lettrine[lines=2, lhang=.2, loversize=.25, lraise=0.05, findent=0.1em,
nindent=0em]{W}{}ir Materieforscher unterscheiden verschiedene Schadklassen von Materialien. Staub gehört zur höchten Klasse, den ganzschädlichen Materialien. Eine Klasse darunter befindet sich beispielsweise Plutonium, dass zur Klasse der zweckschädlichen Materialien gehört. Die Schadklasse von Staub ist für den Wissenschaftler damit eine Höhere als die des Plutoniums.  

Erlauben Sie mir bitte, dies als Anlass zu verstehen, der Öffentlichkeit dieses Material etwas Näher zu bringen. 

<<<<<<< HEAD
Staub entsteht ganz allgemein an der Staubquelle in Form von  Staub-Nucleii. Zu unterscheiden ist der Hausstaub, \textit{pulvis domesticus}, kurz PD, von Industrie- und anderen Stäuben. Nach dem Ausschüttungsvolumen gerechnet ist die wichtigste PD-Quelle Hypotrichose und andere Ursachen von Haarausfall. Hausstaub ist über den Umweg Haarausfall daher mit dem Alter korreliert, woher auch der Ausdruck, etwas sei verstaubt, also alt, kommt. Der unangenehme Geruch alter Menschen hat übrigens mit deren Staubentwicklung nichts zu tun, dies ist eine mittlerweile widerlegte Hypothese, der aus den 60er Jahren. Nuere Befunde aus der experimentellen Staubforschung scheinen der Alterskorreliertheit der Menge emitierter Abwurfhaare in Gramm allerdings in Frage zu stellen. Ein interdiszipliniäres Forschungsprojekt im Auftrag des Verbands der Coiffeure und Frisösen kommt zu dem Ergebniss, dass die volle Haarpracht der Jugend eine grössere PD-Quelle darzustellen scheint, als schüttere Silberzwiebeln. Auch erlebt der Drang zur raschen Wiederaufhaarung durch haarbildenes Gewebe schon im Pre-Best-Ager-Alter seinen Höhepunkt. Auf Haar von liebenswerten mitwohnenden Hunden soll aus Platzgründen nicht eingegangen werden, auf Katzen möchte ich aus Neigung nicht eingehen.
=======
Staub entsteht ganz allgemein an der Staubquelle in Form von  Staub-Nucleii. Zu unterscheiden ist der Hausstaub, \itshape{pulvis domesticus}, kurz PD, von Industrie- und anderen Stäuben. Nach dem Ausschüttungsvolumen gerechnet ist die wichtigste PD-Quelle Hypotrichose und andere Ursachen von Haarausfall. Hausstaub ist daher mit dem Alter korreliert, woher auch der Ausdruck, etwas sei verstaubt, also alt ist, kommt. Der unangenehme Geruch alter Menschen hat übrigens mit deren Staubentwicklung nichts zu tun, dies ist eine mittlerweile widerlegte Hypothese. Neuere Befunde aus der experimentellen Staubforschung scheinen der Alterskorreliertheit der Menge emitierter Abwurfhaare in Gramm allerdings in Frage zu stellen. Ein interdiszipliniäres Forschungsprojekt im Auftrag des Verbands der Coiffeure und Frisösen kommt zu dem Ergebniss, dass die volle Haarpracht der Jugend eine grössere PD-Quelle darzustellen scheint, als schüttere Silberzwiebeln. Auch erlebt der Drang zur raschen Wiederaufhaarung durch haarbildenes Gewebe schon im Pre-Best-Ager-Alter seinen Höhepunkt. Auf Haar von liebenswerten mitwohnenden Hunden soll aus Platzgründen nicht eingegangen werden, Katzen erwähne ich aus Neigung nicht.
>>>>>>> e34b32287e65e382004e94ab4744d90d699d20e1

Eine weitere typische Staubquelle ist der Abrieb. Wissenschaftlich spricht man von Abrasions-Stäuben. Prinzipiell erfährt jedes Material in Wechselwirkung einen Materialverlust, der Grundlage aller Stäuber. Ein kalbender Gletscher ist eine sehr schöne makroskopische Manifestation dieses Vorgangs.

Im Kontext Haushalt ist das textile Material sicher die Königin der Abrasions-Staub-Quellen. In textilarmen Ökologien kann ein Defizit an Staub zur erwarteten Standardmenge nachgewiesen werden.

Das ist übrigens schon mit den rudimentären Bordmitteln menschlicher Sensorik möglich. Man beobachte nur einmal den Boden einer Sauna, selten wird man dort Staub entdecken können. Ganz anders ist die Situation bei Sitzmöbeln mit Textilbezug. Der Abrieb ist häufig so hoch, dass man davon ausgehen muss, dass eine durchschnittliche, zwölf Jahre alte Couch im Kern einen Staubanteil von über 80\,Prozent hat. Derartige Absitzvorichtungen bestehen nur sehr selten aus reinem Textil, sondern beherbergen zuemeist eine Stützvorrichtung aus Metall, Holz oder anderen Materialen der Klasse der Schwerstbigbaren. 

Da mit steigender Summe der Besitzungen eines Möbels die Wahrscheinlichkeit eines Materialbruches der Stützgestänge rasant steigt, ist verhärteter Altstaub eine wichtige und gewollte statische Hilfe. Die Wahrscheinlichkeit steigt übrigens nicht zuletzt, da Absitzhäufigkeit und Gewichtszunahme des Besitzenden empirisch in dieselbe Richtung fahren. Das höhere Gewicht sorgt allerdings seinerseits wiederum für einen stärkeren Abrieb, wodurch ein gutes Möbelstück über viele Jahrhunderte im Gelichgewicht gehalten werden kann, jedenfalls wenn besitzende Besitzer nicht in dieser Zeit die zulässige Maximalverfettung überquillt. Die entsprechenden Angaben sind in BMI den jeweiligen amtlichen Plaketten auf der Frontseite jedes Möbelstücks gut leserlich graviert.

Ein leider viel zu seltenes Phänomen sind explosionsartige Stauberuptionen. Aus dem Haushalt ist das bekannteste Beispiel das Aufschütteln der Bettdecke. Zur Veranschaulichung: durch das Aufschütteln einer norddeutschen Normsteppdecke durch einen standardisierten und von allen Fehlern befreiten Normmenschen, den sogenannten Micheln, werden im Zeitraum von einer Stunde wesentlich mehr Staubnukleii ausgeblasen, als Glaxien in einem Wasserglas Platz finden! Für Stuabforscher ist dieser Anblick super-duper erhaben. Noch Jahre später lässt sich aus der Beobachtung fliegenden Staubs die exakte Aufschüttelbewegungsabfolge ohne weiteres rekonstruieren. Jedenfalls wenn der Raum nach dem Schüttelwirbel nicht mehr betreten wird, was in grossen Schlossanlagen zur Freude der Wissenschaft den Regelfall darstellt. Alleine im Nebengebäude des Nebenhauses eines kleinen Jagdschlosses ganz in der Nähe unseres Uni-Campus stehen noch genügend Zimmer zur Verfdügung, um die entsprechenden Praktika auch der kommenden Generationen Studierender sicher zu stellen.

Aber zurück zum Lifecylce unseres Themas. Nachdem sich jetzt Staubnucleii gebildet haben, wirken auf diese Kräfte, die eine Bewegung im Raum bewirken. Die Natur der Kräfte kann unterschiedlich und häufig additiv sein und wird hier im Modell als gegebene externe Grösse eingeführt und gilt als gegeben. Jede Kraft erzeugt bekanntermassen eine Gegenkraft, im Falle der Bewegungskraft ist dies die statische Kraft, es kommt zu einer statischen Aufladung der Partikel, die wiederum ein Zusammenballen der Nucleii zu grösseren Einheiten bewirkt. Diese Phase des Lebenszyklus des Staubes wird Transport und Akkumulation, TrAkk, genannt. 

Akkumulation 

Weit verbreitet ist der Irrglauben, Staub sei scheu und bevorzuge versteckte Ecken wie beispielsweise die Fläche unter Betten oder Schränken. Vermutlich geht die unhaltbare Mähr auf die Propaganda spätmittelalterlicher Scheinwisschenschaftlerinnen, volkstümlich Hexen, zurück. Aber wir wollen die Damen nicht auch noch nachträglich auf den akademischen Scheiterhaufen schicken, denn bar jeder Grundlage ist das wirre Gerede der Götzendinerinnen nicht. 

Tatsächlich handelt es sich um einen Scheineffekt, der auf einer Scheinkraft beruht. Diese Scheinkraft bildete in der antiken Staubforscherszene das grösste Rätsel ihrer damals aktuellen Forschung. Eingebttet in die mittelalterliche Säfteleere nahmen sie an, dass es Luftströme sind, die Staub bewegen. Luftpakete prallen aber gerade an Objekten wie Schränken und Betten besonders intensiv ab, Kriechströme, die auch nidere Objekte unterkriechen können, konnten nicht nachgewiesen werden. Untersuchungen mit Gegeteiligem Resultat wurden alle wegen orthografischer Fehler für ungültig erklärt und sind nur noch auf den grossen mongolischen Schwarzmärkten zu erhalten. 

Wie also, etwas salopp formuliert, kommt der Staub dann unter das Bett? Betten heizen sich des Nachts auf, befeuert von der Körperwärme der Bettinsassen. Und Wärme ist ja nichts anderes als eine grössere Bewegung der Teilchen, was wie beschrieben zu einer grösseren Akkumulation von Staub führt. Die Staubverteilung ist üblicherweise im Raume gleichverteilt, die Akkumulationsrate dagegen nicht, weswegen der Staub unter dem Bett eher sichtbar ist, als beispielsweise im Kühlschrank. Häufige exotherme Tätigkeiten im Bett befeuert buchstäblich die Staubbildung.

Die Form der Staubballen folgt eigenen Gesetzen, mit denen sich ein grosser Teil der mathematischen Modullierungstheorie beschäftigt. Der interessierte Leser sei auf die Fülle entsprechender Literatur verwiesen, knappe Einführungen finden sich in den Lehrbüchern der Mathematik für die Grundschule. Eine typische Form ist die Staubwalze, wobei es wichtig ist, zwischen rechts- und linksdrehend zu unterscheiden. Sehr viel seltener die Marienform, die, wenn entstanden gerade in spanischsprachigen Ländern häufig tumultartige Reaktionen beim Staubpublikum auslöst. So manche Marienerscheinung kann durch die Staubforschung erklärt werden. Eine App zur korrekten Bestimmung der Staubform ist in Vorbereitung, bis dahin nutze man bitte die Telefon-Servicenummer.

Prinzipiell funktionier Akkumulation immer sehr ähnlich, völlig unabhängig von der Staubquelle. Nur die Luftfeuchtigkeit scheint eine Rolle zu spielen, jedenfalls ist der Formenreichtum mariner Stäube doch eher beschränkt. Für alle Stäube gilt, dass es in einem ersten Schritt zu einer Verkettung der Nucleii zu Stapelfasern kommt, das nur sehr kurzlebige Zwischenprodukt wird Garn genannt. Die Verkettung erfolgt durch Quantenverschränkung ganzer Nucleii, wobei Quarks und Leptonen ausgetauscht werdn, dass die Schwarte kracht. Das Garn verspinnt sich dann durch den Spin der Quarks zu grösseren Strukturen, immer in der Sequenz: einstechen, umschlagen, durchziehen und ablegen. Diese Sequenz wird geloopt.

Ein spannender Effekt hierbei ist das völlige Verschwinden jeglicher Farbpigmente der Nucleii. Staub ist immer grau und hat damit keine Farbe, wie ihnen jedes achtjährige Kind aus einer Familie, in der auf Bildung Wert gelegt wird, sagen kann. Da aber alles eine Farbe hat, haben muss, ist Staub nicht! Staub gehört damit nicht zu den Dingen, sondern ist eine Materialität eigener Art, sui generis, wie wir sagen. Der Grund ist einmal mehr bei der Akkumulation zu suchen. Die dünne Schicht der Staubnucleii ist zunächst zweidimensional, räumlich also gar nicht existent. Daher ist Staub auch gewichtslos, wie sie selber leicht nachprüfen können. Vergleichen sie einmal eine Staubrolle mit einem modernen Rennrad aus dem Hochpreissegment. Sie werden merken um viel das Rad schwerer ist, obwohl eigentlich unglaublich leicht! 

Durch Akkumulation erhöht sich die Dimensionalität Schritt für Schritt von der Zweiheit als Ausgangspalteau auf einen fraktalen Wert von maximal 2.3. Dieser entspricht merkwürdigerweise genau der Dimensionalität der Alpen! Umgangssprachlich wird diese mystische Verwandlung aus dem Nichts in etwas fast Seiendes auch als \textit{aus dem Staub erheben} bezeichnet. Dieser Gedanke ist es, den die meisten meiner Kolleginnen und Kollegen, so auch mich, Tag für Tag antreibt, sich mit dem Phänomen zu beschäftigen.

Das Ende des Lebenszyklus von Staub bildet dessen Vernichtung durch Putzen. Diese Vernichtung ist aber nur scheinbar. Tatsächlich kann Staub nur bei sehr hohen Temperaturen vernichtet werden, die lediglich im absoluten Zentrum von Vulkanen auf der Erde vorkommen und meist nur durch gollumbewachte Höhlen zu erreichen sind. Ausserdem scheint es Hinweise zu geben, dass der Einstatz sehr stark potenzierter homöopathischer Säuren einen gewissen Reduktionseffekt hervorrufen kann. Da diese aber nur an Schurkenstaaten mit Chemiewaffenschein abgegeben werden, ist keine praktische Relevanz gegeben und die Forschungslage dünn.

Da eine Vernichtung von Staub im Hausgebrauch unmöglich ist, empfiehlt sich eine Dislotion. Im Fachhandel werden Geräte vorrätig gehalten, die sich den Horror Vacui als Grundprinzip der Natur zu Nutzen machen. Luft wird aus einem Behälter evakuiert, die entstehende Leere will sich füllen und nimmt statt der offensichtlich unzuverlässigen Luft lieber Staub. Leider erzeugt diese Leere auch immer ein lautes Geräusch, den sogenannten Schrei der Leere. Dieser wird immer als hochgradig störend empfunden, da die Leere der Natur der Welt widerspricht. Der Schrei der Leere erzeugt allgemein Wut. Beim Benutzenden des Gerätes muss der Impuls zu fliehen unterdrückt werden. Dies erfolgt durch einen quasireligiösen Ritus. Die zu behandelnde Fläche wird dabei vollständig abgeschritten, der Blick muss demütig in Richtung Boden gesenkt bleiben. Das funktioniert ganz unabhängig von der Benutzung der Gerätschaft, sollten sie sich also einmal dem Schrei der Leere durch ihre Nachbarn ausgesetzt fühlen, senken sie den Blick und schreiten sehr ruhig und massvoll ihre Wohnung ab, Blick immer auf die Füsse.

Staub kann auch auf natürliche Art vernichtet oder genauer transformiert werden. Wenn eine Staubschicht eine kritische Dichte erreicht hat, bildet sich eine Staubschicht. Der Begriff Schicht ist der Geologie entlehnt, die allgemein als Schwesterdisziplin der Staubwissenschaft verstanden werden kann. An kleineren Universitäten ohne eigenen staubwissenschaftlichen Studiengang, ist die Staubwisschenschaft häufig eine Vertiefungsrichtung der Geologie, ähnlich der Paläontologie und der Zahnsteinkunde. Methodisch konnten viele Methoden der Geologie auf die Staubwissenschaft adaptiert werden.

Ist jetzt also die Staubschicht kritisch dicht, man spricht auch von Sedimentierungsablagerung, kommt es zu Tektonik. Dabei gleitet die Staubschicht auf dem Trägermaterial durch den Raum und wird letztendlich unter den Tepich subduziert. Die Gewichtsauflage durch Betritt des Tepichs bewirkt eine weitere Verdichtung, ausserdem verbindet sich der Staub mit dem Teppich. Ähnlich wie das kochende Erdinnere führen Bodenheizungen weitere Energie zu. In der Staub-Teppich-Fazies kann der Staub im Laufe der Jahrmillionen auskristallisieren. In Teppichregionen mit hohem Flüssigkeitsverschütt bilden sich Stalagtiten, oft von beeindruckender Grösse. Je nach Flüssigkeit kann es darüber hinaus zu Farbeinträgen kommen, Rotwein färbt beispielsweise blau und der enthaltene Weinstein klumpt sich zu Rubinen. Minenkinder aus Kriesenregionen bauen diese dann ab, wo sie dann bei der Produktion von monarchischem Ziergerät (Kronen usw.) oder als Grundlage für Rubinlaser in der rohstoffgierigen Laserpointerindustrie Verwendung finden.

Der letzte Abschnitt soll die Wirkung des Staubes auf den Menschen behandeln. Staub ist als nur fast existentes Material eigentlich völlig harmlos und unschädlich, obwohl wie geschrieben der höchsten Gefahrenklasse angehörig. Trotzdem, teilweise auch gerade deswegen reagieren viele Menschen sehr intensiv auf Staub. Auch dies ist ein Phänomen, welches bisher unverstanden ist. Die Ursache scheint aber genetisch bedingt zu sein. Männer sind im Allgemeinen sehr viel weniger betroffen als Frauen und empfinden ihn erst in sehr viel höheren Konzentrationen als störend. Alternativ zur genetischen Disposition kommen aus der Genderforschung neue Impulse, den Unterschied zu verstehen, vermutlich lässt sich das Ganze auch mit dem generischen Maskulin wegdiskutieren. 

Die Wirkung von Staub auf Gesellschaft ist noch in anderen Punkten mannigfalitig. Die Kulturwissenschft als verstaubteste Wissenschaft liefert eine Fülle von Beispielen. Es lässt sich beispielsweise zeigen, dass Regionen zu Füssen von Gebirgen, in denen viel Staub angeweht wird, auch eine besondere Putzkultur hervorgebracht haben. Eingebohrene betrachten den Akt des Putzen oft als etwas Heiliges und sehen darin die letztendliche Bestimmung des Seins. Man spricht von Calvinistischen Kulturen (ein Calviner war ein mittelhochkeltischer Besen von fünf bis sechs Metern Breite, der aus sogeannten Reeddächern gewoben wurden.)



Anlässlich des letzten World Particulate Dust Palavers in Kylmäluoma, Finnland, wurde ich von Entscheidungsträgern aus Politik und Showbusiness gefragt, ob mit einer Verstaubung der Welt zu rechnen sei und welche Massnahmen ergriffen werden können. Die einfache Antwort ist ja, damit ist zu rechnen und nein, es lässt sich nichts dagegen unternehmen. Wenn man bedenkt, dass es selbst mittelkomplexe Stäube schaffen, das Genom von Haustieren dahingehend zu verändern, mehr Staub zu produzieren, ist mit einer endgültigen Verstaubung der Welt in spätestens zwei Jahren zu rechnen, auf abgelegnen Atollen mag es noch etwas länger dauern. Genau lässt sich das nicht präzisieren, die Supercluster von Supercomputer, die mit der Modellierung beschäftigt sind, sind leider bereits jetzt endgültig verstaubt.

 \vfill
