\thispagestyle{empty}
\begin{center}
\includegraphics[width=\textwidth]{./bilder/schwert.png}
\end{center}
\vspace*{\fill}
%{\Huge\color{farbe}\hfill{\ttfamily{Fangen}}}
{\centering\fontsize{50}{48} \color{farbe}\sffamily{Laura und das Erwachen ihrer Schildkröte}\par}
{\centering\fontsize{20}{28} \color{farbe}\sffamily{Von Paula alleine geschrieben}\par}
\addcontentsline{toc}{chapter}{Laura und das Erwachen ihrer Schildkröte}
\newpage
%%%%%%%%%%%%%%%%%%%%%%%%%%%%%%%%%%%%%%%%%%%%%%%%%%%%%%%%%%%%%%%%%%%%%%%%%%%%%%%
\lettrine[lines=3, lhang=.2, loversize=.25, lraise=0.05, findent=0.1em,
nindent=0em]{L}{aura} ein 14-jähriges Mädchen, hat eine Schildkröte namens Naila. Laura sagt immer: \enquote{Naila ist die schönste Schildkröte auf der Welt}.Seid 5 Monaten ist Naila schon unter der Erde. Denn sie hält Winterschlaf,Laura zählt jeden Tag das sie beim nächsten Mal genau weiss wie lang es noch dauert, bis sie raus kommt. 

Plötzlich hat Laura ein merkwürdiges Gefühl. Wie von Zauberhand gesteuert rennt sie zu Neilas Käfig. Da! Naila ist draussen. Laura ruft \enquote{Hurra!} Am nächsten Tag ruft sie ihre beste Freundin an. Sie heisst Klara. Beide machen ab. Sie waren so glücklich, dass sie dreimal in der Woche zusammen spielten. Laura, Klara und Naila waren die drei besten Freundinnen für immer und sie erlebten noch viele weitere Abenteuer.

An einem Morgen wachte Laura auf. Sie bemerkte nicht dass ein Zettel auf ihrem Pult lag. Sie frühstückte ihr Lieblingsessen, Spiegelei und Toast mit Nutella. Laura sagte \enquote{Mhhh! Lecker!} Sie ging später zu Klara. Klara sass auf ihrem Bett. Sie sagte: 

\enquote{Warum bist du so glücklich?} Laura sagte: 

\enquote{Warum bist du so traurig?} Sie sagte:

\enquote{Weil Naila sehr krank ist natürlich}

Laura schrie \enquote{Was?}

\enquote{Ja, deine Mutter hat mich vorhin angerufen.}

\enquote{Aber warum weiss ich nichts davon?}

\enquote{Hat es dir diene mutter nicht gesagt?}

Laur sagte: \enquote{Nein.}

Sie gingen zu Laura. 

\enquote{Da}, sagte Klara, \enquote{ein Brief von Deiner Mutter.} Luara sagte

\enquote{Aha! Deshalb habe ich nichts davon gewusst!}

Sie riefen sofort Lauras Mutter an. Sie sagte:

\enquote{Ich war so beschäftigt, dass ich vergessen hatte die Ärztin anzurufen. Könnt ihr die Ärztin anrufen?}

Beide rufen:
\enquote{Ja!}

Kurz darauf ist die Ärztin da. Sie sagte:

\enquote{Ihr müsst ihr jeden Tag diese Medikamente geben. Dann wird sie schon bald wieder gesund.}

Laura und Klara taten das und nach einer Woche war Naila wieder gesund. \hfill \pgfornament[color=farbe,height=.5cm]{3}

