\thispagestyle{empty}
\begin{center}
\includegraphics[width=\textwidth]{./bilder/fangen.png}
\end{center}
\vspace*{\fill}
%{\Huge\color{farbe}\hfill{\ttfamily{Fangen}}}
{\centering\fontsize{50}{48} \color{farbe}\sffamily{Fangen}\par}
\addcontentsline{toc}{chapter}{Fangen}
\newpage
%%%%%%%%%%%%%%%%%%%%%%%%%%%%%%%%%%%%%%%%%%%%%%%%%%%%%%%%%%%%%%%%%%%%%%%%%%%%%%%
\lettrine[lines=3, lhang=.2, loversize=.25, lraise=0.05, findent=0.1em,
nindent=0em]{T}{obias} hatte zwei Probleme an diesem Nachmittag. Das erste Problem war eines, ganz nach seinem Geschmack. Jeden Abend hatte er Diskussionen mit seiner Mutter, warum er schon ins Bett musst. Er wollte ihr ganz sachlich erklären, dass Schlafen aber gar nicht nötig sei. Tobias vertraute schon immer der Wissenschaft. Also sass er vor seinem Rechner und suchte das Internet nach Erklärungen ab, warum Menschen – und offensichtlich auch Tiere, schlafen. Verwirrenderweise schien es aber keine echte Erklärung zu geben. Gut, man stirbt, wenn man nicht schlafen kann und wird vorher wahnsinnig. Beides keine Ziele, die Tobias anstrebte. Aber auf die Frage Warum? Schien es noch keine Antwort zu geben. Sehr verwirrend. Dabei gab es doch für alles eine Erklärung. Am meisten befriedigten Tobias die überraschenden. Das Boote, wenn sie auf einem Fluss treiben, schneller sind als das Wasser, zum Beispiel. Als er recherchiert hat, was ein Regenbogen eigentlich ist, war er verblüfft, dass nie jemand den selben Regenbogen sieht und dass es Regebogen gibt, die man nur von Flugzeugen aus sehen kann und kreisrund sind. Aber beim Schlafen kam er nicht weiter. Jedenfalls im Augenblick nicht, denn das zweite Problem machte sich deutlich bemerkbar.


\enquote{Tobias, die Sonne scheint und du sitzt seit Stunden nur vor dem Computer. Du musst mal raus, dich bewegen, mit den anderen spielen!}, sagte das Problem. Dabei fand Tobias das, was seine Mutter unter Spielen verstand, meist sinnlos. Umherrennen ohne Sinn? Wozu? Wenn er vor dem Computer sass oder ein Buch las, wusste er danach mehr als vorher. Da war etwas zu lernen und was gab es schöneres, als neue Dinge zu lernen? Immer wenn er mit einem neuen Thema anfing, taten sich Welten auf, in denen er versinken konnte. Und was gibt es für grössere Abenteuer, als Neues zu entdecken? Mit dem Schiff über das Meer und Amerika finden? Durchs Mikroskop sehen und Bakterien entdecken? Nur dank der Wissenschaft hatte sich die Welt so sehr vergrössert! Wieso also Zeit verschwenden und kreischend durch den Innenhof rennen?

\enquote{Sieh mal, die meisten Kinder aus dem Block sind draussen und spielen Fangen. Nur der Ältere von Familie Demirci, wie heisst der noch gleich? ist krank. Dem sollst du ja auch noch die Hausaufgaben bringen, also los.}

Kurz überlegte Tobias, ob er auch schnell ein Hüsteln vortäuschen sollte, aber das wäre jetzt zu plump. Also setzte er zu einer Erklärung an, der Mama nicht widersprechen konnte. Gott sei Dank hatte er die Wissenschaft, die kann einem immer helfen. Ein unschlagbares Argument war eben ein unschlagbares Argument, da konnte Mama noch so viel älter sein, das spielt dann keine Rolle mehr.

\enquote{Liebe Mama,} Ein Stöhnen seiner Mutter unterbrach ihn. Etwas sagte Tobias, dass seine ständigen Diskussionen und Belehrungen vielleicht für andere auch anstrengend sein könnten, aber die Wahrheit durfte keine Rücksichten nehmen.

\enquote{Was ist es denn diesmal mein Schatz?} Na gut, wenigstens war sie bereit zuzuhören.

\enquote{Fangen zu spielen ist eine freudlose Sache.}, fing er also an. Das Wort freudlos hatte er eben erst aufgeschnappt und hoffte, dass es an dieser Stelle passte und seinem Argument mehr Würde und Wissenschaftlichkeit verlieh.

\enquote{Es muss nämlich immer folgendes passieren: Ein Spieler} --er sagte bewusst nicht Kind-- \enquote{beginnt mit Fangen. Es rennt so lange anderen Spielern hinterher, bis er\dots}

\enquote{\dots oder sie}, unterbrach ihn seine Mutter, auch sie hatte Spass daran, ihn zu korrigieren, wenn sie schon einmal die Gelegenheit hatte,

\enquote{Ähm ja, also er oder sie jemanden erwischt. Diese Person muss eine sein, die langsamer ist als die Fängerin oder der Fänger.}

Den letzten Teil betonte Tobias überdeutlich.

\enquote{Diese jetzt neu zum Fangen verpflichtete Person kann wiederum nur jemanden fangen, der oder die seinerseits oder ihrerseits – Mama das nervt, ich sage jetzt nur noch sie – die wiederum selbst langsamer ist. Wenn letztendlich die langsamste Spielerin, was übrigens von Anfang an der Fall sein kann, an der Reihe ist, ist das Spiel praktisch vorbei. Die erwischt niemanden mehr.}

Die Mutter rieb sich sehr theatralisch das Kinn, so als würde sie lange und gründlich nachdenken.

\enquote{Mein lieber Tobias} liess sie sich auf den Tonfall ihres Sohnes ein. \enquote{Ich sehe das Zwingende in deinem Argument, es klingt absolut wissenschaftlich wie du sagen würdest. Aber so gut dein Argument auch ist, wenn ich aus dem Fenster blicke, sehe ich, dass die anderen Kinder fast jeden Tag Fangen spielen. Es scheint ihnen nicht langweilig zu werden.. Ich selbst habe es als Kind auch sehr oft gespielt. Wenn du Recht haben würdest, wäre es doch schon längst ausgestorben, gibst du mir da Recht?}

Den Punkt musst Tobias abgeben. Da hatte ihn seine Mutter geschlagen. Sein triumphierender Blick war von hängenden Schultern abgelöst worden. Sie hatte völlig Recht mit dem, was sie sagte. In seiner Überlegung musste ein Fehler sein. Aber wo? Er dachte an Jakob Degen, der 1807 als einer der ersten nicht nur eine Flugmaschine erdacht hatte, sondern sie auch ausprobiert hat. Oder Jane Goodall, die um zu beweisen, dass Menschen und Affen verwandt sind, jahrzehntelang mit Schimpansen im Urwald gelebt hat. Es half nichts, er musste selber auch ausprobieren.

\enquote{Ich muss los Mama!}, rief Tobias also zu seiner schmunzelnden Mutter, \enquote{die Wissenschaft ruft.}, nahm seine Jacke und verschwand nach draussen.

Als es schon lange dunkel war und Tobias und seine Mutter butterbrotschmierend am Tisch sassen, fragte sie:

\enquote{Und, hast du deine Theorie aufgegeben? Es scheint dir ja ziemlichen Spass gemacht zu haben.}

Mist. Die Theorie hatte er ganz vergessen. Anfangs hatte er noch beobachten wollen, aber dann war gar keine Zeit mehr dafür gewesen. Bei ungefähr neun Kindern war es schwierig genug zu merken, wer gerade Fänger war. Aber die Logik seiner Theorie überzeugte ihn immer noch, warum stimmte sie bloss nicht. Also zuckte er nur mit den Schultern.

\enquote{Keine Ahnung}, stammelte er. Tobias Mutter, die zwar nicht immer, aber in diesem Fall doch Spass daran hatte, zusammen mit ihm ein bisschen die Gedanken treiben zu lassen, hatte einen Ansatz:

\enquote{Ich glaube der Fehler liegt darin, dass es so etwas wie das schnellste Kind.}, \enquote{Spieler} korrigierte Tobias, \enquote{Die Erklärung muss für alle gelten, nicht nur Kinder}. Fast hätte die Mutter an dieser Stelle doch die Lust verloren, machte aber weiter.

\enquote{Den schnellsten Spieler von mir aus, nicht gibt. Kannst du dich an Olympia letzten Sommer erinnern?} Konnte Tobias natürlich, auch wenn er nie verstehen konnte, was seine Mutter gereizt hatte, stundenlang vor dem Fernseher zu sitzen und anderen beim Sport zuzusehen.


\enquote{Dort gibt es alleine beim Um-die-Wette-Laufen viele verschiedene Disziplinen: 100m, 200m, 400m, 800m, 1500m, 10000m und Marathon}, \enquote{42.195km}, warf Tobias kauend ein. \enquote{Hinzu kommen Hürdenläufe in verschiedener Länge, Hindernislauf und Staffel. In allen Disziplinen gibt es jeweils eine Frau und einen Mann, die die schnellsten sind. Das bedeutet aber nicht, dass sie in den anderen Disziplinen auch gut sind. Von Ursain Bolt, einem berühmter Sprinter, habe ich mal gelesen, dass er schon auf einer Strecke von 1000m nicht einmal zu den besten einer normalen Schule gehören würde.}

Das leuchtete Tobias sofort ein. Da konnte der Fehler liegen. Natürlich, alle Spieler haben unterschiedliche Laufstärken. Und nicht nur bei der Distanz. Er zum Beispiel hatte sich oft retten können, weil er sehr schnell einen Haken schlagen und die Richtung wechseln konnte. Andere hatten ein gutes Gefühl dafür, wo sie möglichst unauffällig stehen konnten. Geradezu aufgeregt listete er seiner Mutter Dinge auf, die ihm jetzt erst auffielen und die bestimmt einen Einfluss auf das Spiel hatten.

\enquote{Genau}, faste seine Mutter den Wasserfall seiner Überlegungen zusammen. \enquote{Sich über solche Sachen Gedanken zu machen, nennt man Taktik. Die wird besonders wichtig bei Spielen oder Sportarten, bei denen gleichzeitig viele miteinander spielen. Neben Kraft, Geschick, Ausdauer und so etwas, spielt es auch eine Rolle, Situationen analysieren zu können.}

Noch abends im Bett musste Tobias lange über das Thema nachdenken. Das machte er gerade lieber als über die Notwendigkeit von Schlaf zu diskutieren. Und so kam es, dass er als Ergebnis seiner Überlegungen, ganz im Sinne der Wissenschaft und nichts weiter, am folgenden Tag seine Mutter bat, ihn doch mal zu einem Probetraining im Fussballverein anzumelden.\hfill \pgfornament[color=farbe,height=.5cm]{3}
\newpage
 

 
